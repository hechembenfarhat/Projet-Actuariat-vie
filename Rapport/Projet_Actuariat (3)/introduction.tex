\chapter*{Introduction générale}
\par La population danoise représente environ 5 750 000 habitants, ce qui équivaut à 1,1$\%$ de la
population européenne. Avec 5 447 000 habitants en 2007 et 5 560000 en 2011, elle a un taux de croissance de 0,7$\%$ et augmente plus rapidement que dans le reste de l’Union européenne. Comme dans l’ensemble de la région nordique, à un vieillissement de la population plus élevé que dans les autres pays de l’Union s’ajoute une ouverture importante à l’immigration.

\par Avec 19,43$\%$ de la population ayant 65 ans ou plus, le Danemark est le onzième pays avec la population la plus âgée dans le monde. La part des 65-79 ans est celle qui a le plus augmenté depuis dix ans, passant de 11.1$\%$ de la population en 2006 à 14.6$\%$. Les projections prévoient une augmentation importante de la classe la plus âgée : la part des plus de 70 ans augmenterait de 64$\%$ entre 2017 et 2050.

\par L’espérance de vie des Danoises à la naissance est la plus faible parmi les treize pays européens pour lesquels on dispose de données en 1999 (c’est-à-dire sans l’Italie et la Grèce) et les résultats en termes d’espérance de vie masculine sont relativement peu satisfaisants. On retrouve le même phénomène lorsqu’on analyse l’espérance de vie à 65 ans, à la fois pour les hommes et pour les femmes.

\par Le Danemark se singularise ainsi des autres pays européens dans la mesure où, entre 1980 et 1995, l’espérance de vie à la naissance des femmes n’a progressé que de 6 mois.
Le taux de mortalité infantile et son evolution depuis 1980 confirment ces résultats sanitaires plutôt décevants ; en
effet, alors que le taux de mortalité infantile était relativement faible au Danemark en
1980, les progrès réalisés y ont été beaucoup plus lents que dans les autres pays européens 

\par D’autres éléments, liés aux comportements et aux modes de vie, sont également intéressants à noter. Le taux de mortalité par cancer est, par exemple, largement supérieur à celui observé dans les autres
pays (222 décès sur 100 000, contre 159 en Finlande et en Suède), mais l’incidencedes cancers y est aussi supérieure (561 cas pour 100 000, contre 420 en Finlande et 483 en Suède). 
Une étude parue dans le British Medical Journal en 20003 met en évidence la forte mortalité associée au tabac chez les femmes. Si, durant les années cinquante, le tabac était responsable de
1 $\%$ des décès des femmes entre 55 et 84 ans, en 1990 ce chiffre s’élevait à un quart des décès.
Les chiffres publiés par l’OMS(Organisation mondiale de la santé) sur les causes de décès corroborent cette analyse.
Le Danemark arrive en tête des pays européens en ce qui concerne les décès liés à l’alcool pour les femmes et deuxième pour les décès liés au tabac.