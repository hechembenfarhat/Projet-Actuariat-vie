\chapter{Contexte général du projet}

\section*{Introduction}
Ce chapitre est consacré à la présentation générale du projet. Nous commençons par l'explication des motivations pour entreprendre ce projet. Ensuite, nous focalisons sur les objectifs de notre sujet et la solution proposée après avoir discuté la problématique à résoudre.
\section{Motivation}
Les Danois vivent plus longtemps que jamais. En 2016, l’espérance de vie à la naissance des Danois a atteint 78,8 ans pour les hommes et 82,8 ans pour les femmes. Depuis 2000, les hommes ont ainsi gagné 4,5 ans d’espérance de vie et les femmes 3,8 ans, grâce notamment à la baisse de la mortalité. Les gains obtenus par les femmes sont moins rapides que ceux des hommes et l’écart entre les sexes se resserre : de 4 ans et 7 mois en 2000, il est passé à 4 ans en 2016.

L’espérance de vie à 67 ans progresse également. Selon les dernières estimations de l’agence nationale Danmarks Statistik, les hommes danois âgés de 67 peuvent espérer de vivre encore 16,6 ans et les femmes 19,1 ans. Cela constitue depuis 2000 un gain de 2,9 ans pour les hommes et de 2,4 ans pour les femmes, donc à nouveau un rétrécissement de l’écart entre les sexes.

C'est pour cela que nous nous intéressons à ce sujet afin d'estimer et de projeter la mortalité de la population danoise (hommes et femmes) pour des âges supérieurs à 50 ans.

\section{Planification du projet}
\subsection{Objectif général}
L’objectif du projet est d'estimer et de prédire le taux de mortalité d'un groupe de danois avec 2 modèles différents: 
\begin{itemize}
    \item[*] Le modèle de Lee-Carter.
    \item[*] Le modèle de Cairns Blake Dowd (CBD).
\end{itemize}
\subsection{Planification prévisionnelle du projet}
Pour assurer le bon déroulement et l'avancement du travail on a mis en place une planification qui s'étale en 3 étapes selon les besoins spécifiques décrits précédemment.
\begin{itemize}
\item[*] \textit{Du 6-9 Mai } Exploitation et compréhension des données. 
    \item[*] \textit{Du 10-16 Mai} Compréhension des notations et préliminaires actuariels.
    \item[*] \textit{Du 17-30 Mai } Codage sous R et Rédaction du rapport en parallèle.
    
\end{itemize}
\section*{Conclusion}
Dans ce qui précède, nous avons abordées le contexte général de ce projet et plus
précisément l’environnement de réalisation. De même, nous avons précisées les objectifs du projet par une planification prévisionnelle qui nous a permis d’assurer le bon déroulement du travail et de garantir le respect des délais.

