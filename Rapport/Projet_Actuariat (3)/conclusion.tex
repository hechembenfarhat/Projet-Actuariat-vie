\chapter*{Conclusion générale}
L’objectif de ce sujet était d’estimer et de projeter le taux de mortalité avec  les methodes Lee-Carter et CBD. Notre étude a été effectuer sur une population venant exactement de « Danemark » qui est une population assez spéciale dans le monde par ce qu’elle est caractérisée par plusieurs guerres d’où la baisse de taux de mortalité. Durant la période du projet on a essayé de traiter le sujet en répondant à des questions bien spécifique permettant de résoudre des problèmes d’actualités, on a estimé les taux de mortalité des hommes et des femmes en utilisant le modèle de Lee-carter et CBD qui sont Des références dans le domaine de prévision de mortalité qu’on a déjà estimé ses paramètres; à la fin notre groupe a essayé de comparer les deux modèles. 
Durant l'exploitation du taux de mortalité de la population danoise pour la période allant de 1876 jusqu'à 2019 nous avons constaté quelques instabilités pour des durées ou le pays a vécu des guerres, soit civile ou externe, cela avait un impact sur la variation du taux de mortalité de la Danemark.
Pour notre étude de cas et puisqu'on s'interesse à la population agés de plus que 50 ans nous avons opté à l'utilisation du modèle Cairn Blake Dowd qui est plus performant pour les cohortes possédant un age plutôt avancé. Selon la documentation fournie par la vigniette de StMoMo, il est conçu d'utiliser le modèle CBD pour la projection des taux de mortalité s'il s'agit d'une population agé et la durée de projection n'est pas assez large.
Dans notre cas nous avons constaté que le modèle CBD a présenté un intervale de confiance moin large que celui du modèle de Lee-Carter, maiscet intervalle couvre une période plus vaste au niveau du CBD que celui du Lee-Carter.
Pour conclure le projet a été très bénéfique pour tous les membres de l’équipe et nous a permis de découvrir de plus proche le monde d’actuariat vie en travaillant sur des données réelles.